\documentclass[10pt]{article}

\usepackage{amsmath, amssymb, amsthm, newtxtext, physics}
\usepackage{float}

\title{\textbf{The Hausdorff Measure}}
\date{}
\usepackage[margins = 0.25in]{geometry}
\theoremstyle{plain} 
\newtheorem{definition}{Definition}
\newtheorem{theorem}{Theorem}
\newtheorem{example}{Example}
\newtheorem{observation}{Observation}
\newtheorem{matlab}{MATLAB}
\DeclareMathOperator{\diam}{diam}
\DeclareMathOperator{\dist}{dist}
\begin{document}
	\maketitle 
	
In geometric measure theory, we like to work with outer measures so much that we just call them measures. 
	\begin{definition}
		A(n) (outer) \textbf{\textit{measure}} $\mu$ on $\mathbb{R}^n$ is a function $\mu: \mathcal{P}(\mathbb{R}^n) \to [0, + \infty]$ such that, for $\{A_i\}_{i \in \mathbb{N}}$ a countable collection of subsets of $\mathbb{R}^n$ and any $A \subseteq \bigcup_{i \in \mathbb{N}} A_i$, we have $$\mu(A) \leq \sum_{i \in \mathbb{N}} \mu(A_i).$$ A set $A \subseteq \mathbb{R}^n$ is \textbf{\textit{measurable}} if, for all $E \subset \mathbb{R}^n$, $$\mu(E \cap A) + \mu(E \cap A^C) = \mu(E).$$
	\end{definition}

	
\subsection*{The Hausdorff Measure}
	Some notation:	
		\begin{itemize}
			\item The \textbf{\textit{diameter}} of a set $S$ is denoted by $\diam(S)$ and is given by the following formula: $$\diam(S) = \sup \{|x - y| : x, y \in S\}.$$ 
			
			\item The Lebesgue measure of the closed unit ball $\mathbb{B}^m(0, 1) \subseteq \mathbb{R}^m$ is denoted by $\alpha_m$ and is given by the following formula $$\alpha_m = \frac{\pi^{m/2}}{\Gamma\left(\frac{m}{2} + 1\right)}.$$ 
		\end{itemize}
	\begin{definition}
		For any $A \subseteq \mathbb{R}^n$ define the $(\delta, m)$-\textbf{\textit{Hausdorff measure}} $\mathcal{H}_\delta^m(A)$ as $$\mathcal{H}_\delta^m(A) = \inf_{\substack{A \subset \bigcup S_j \\ \diam(S_j) \leq \delta}} \sum \alpha_m \left(\frac{\diam(S_j)}{2}\right)^m.$$ The \textbf{\textit{$m$-dimensional Hausdorff measure}} is defined as $\mathcal{H}^m(A) = \lim_{\delta \to 0} \mathcal{H}_{\delta}^m(A)$. 
	\end{definition}
	
	\newpage 
	\begin{observation} ~
		Let $\mathcal{H}^m$ be the $m$-dimensional Hausdorff measure over $\mathbb{R}^n$. 
			\begin{itemize}
				\item[(1)] $\mathcal{H}^m$ is countably subadditive. 
				
				\item[(2)] All Borel sets of $\mathbb{R}^n$ are $\mathcal{H}^m$-measurable. 
				
				\item[(3)] For all $A \subset \mathbb{R}^n$ there exists a Borel subset $B \subset \mathbb{R}^n$ such that $$\mathcal{H}^m(A) = \mathcal{H}^m(B).$$
			\end{itemize}
	\end{observation}

\noindent Note: Conditions (2) and (3) give that $\mathcal{H}^m$ is a \textbf{\textit{Borel regular measure}}. 
	\begin{proof} ~
		\begin{itemize}
			\item[(1)]  Let $\{A_i\}_{i \in \mathbb{N}}$ be a countable collection of subsets of $\mathbb{R}^n$, and $A$ any subset of $\bigcup_{i \in \mathbb{N}} A_i$. Then,
				\begin{align*}
					\mathcal{H}_\delta^m(A) &= \inf_{\substack{A \subseteq \bigcup S_j \\ \diam(S_j) \leq \delta}} \sum  \alpha_m \left(\frac{\diam(S_j)}{2}\right)^m \\
					\mathcal{H}_\delta^m(\bigcup A_i) &= \inf_{\substack{\bigcup A_i\subseteq \bigcup S_j \\ \diam(S_j) \leq \delta}} \sum  \alpha_m \left(\frac{\diam(S_j)}{2}\right)^m  
				\end{align*}
			Now, any countable covering of $\bigcup_{i \in \mathbb{N}} A_i$ is also countable covering of $A$. Therefore, 
				$$\left\{\{S_j\}_{j \in \mathbb{N}} : \bigcup_{i \in \mathbb{N}} A_i \subseteq \bigcup_{j \in \mathbb{N}} S_j \text{ and } \diam(S_j) \leq \delta\right\} \subseteq \left\{\{S_j\}_{j \in \mathbb{N}} :A \subseteq \bigcup_{j \in \mathbb{N}} S_j \text{ and } \diam(S_j) \leq \delta\right\}.$$ Recall that $A \subseteq B \implies \inf A \geq \inf B$. Hence, $$\mathcal{H}_\delta^m\left(\bigcup A_i\right) \geq \mathcal{H}_\delta^m(A).$$ Passing to the limit as $\delta \to 0$ gives the result. 
			
			\item[(2)] Note that all Borel sets are measurable if and only if Cartheodory's criterion holds:  $$\mathcal{H}^m(A_1 \cup A_2) = \mathcal{H}^m(A_1) + \mathcal{H}^m(A_2)$$ for all $A_1, A_2 \subseteq \mathbb{R}^n$ with $$\dist(A_1, A_2) := \inf\{|x - y| : x \in A_1, y \in A_2\} > 0.$$ So, we check Cartheodory's criterion for $\mathcal{H}^m$ and pack our bags. So, let $A_1, A_2 \subseteq \mathbb{R}^n$ with $\dist(A_1, A_2) > 0$. Since we already know that $$\mathcal{H}^m(A_1 \cup A_2) \leq \mathcal{H}^m(A_1) + \mathcal{H}^m(A_2)$$ by countable (which implies finite) subadditivity, it suffices to prove $$\mathcal{H}^m(A_1) + \mathcal{H}^m(A_2) \leq \mathcal{H}^m(A_1 \cup A_2).$$ We have,
				\begin{align*}
					\mathcal{H}_\delta^m(A_1) + \mathcal{H}_\delta^m(A_2) &= \inf_{\substack{A_1 \subseteq \bigcup S_j^1 \\ \diam(S_j^1) \leq \delta}} \sum \alpha_m \left(\frac{\diam(S_j^1)}{2}\right)^m + \inf_{\substack{A_2 \subseteq \bigcup S_j^2 \\ \diam(S_k^2) \leq \delta}} \sum \alpha_m \left(\frac{\diam(S_k^2)}{2}\right)^m \\
					&= \inf_{\substack{A_1 \subseteq \bigcup S_j^1, A_2 \subseteq \bigcup S_k^2 \\ \diam(S_j^1) \leq \delta, \diam(S_k^2) \leq \delta}} \sum_{j, k} \alpha_m \left[\left(\frac{\diam(S_j^1)}{2}\right)^m + \left(\frac{\diam(S_k^2)}{2}\right)^m\right] \\
					&\leq \inf_{\substack{A_1 \subseteq \bigcup S_j^1, A_2 \subseteq \bigcup S_k^2 \\ \diam(S_j^1) \leq \delta, \diam(S_k^2) \leq \delta}} \sum_{j, k} \alpha_m  \left(\frac{\diam(S_j^1) + \diam(S_k^2)}{2}\right)^m \\
					&\leq \inf_{\substack{A_1 \subseteq \bigcup S_j^1, A_2 \subseteq \bigcup S_k^2 \\ \diam(S_j^1) \leq \delta, \diam(S_k^2) \leq \delta  \\ S_j^1 \cap S_k^2 = \emptyset}} \sum_{j, k} \alpha_m  \left(\frac{\diam(S_j^1 \cup S_k^2)}{2}\right)^m \\
					&\leq \inf_{\substack{A_1 \subseteq \bigcup S_j^1, A_2 \subseteq \bigcup S_k^2 \\ \boxed{\diam(S_j^1 \cup S_k^2) \leq \delta }  \\ S_j^1 \cap S_k^2 = \emptyset}} \sum_{j, k} \alpha_m  \left(\frac{\diam(S_j^1 \cup S_k^2)}{2}\right)^m \\
				\end{align*}
			For any covering $\{S_j\}_{j \in \mathbb{N}}$ of $A_1 \cup A_2$, observe that by virtue of the fact $\dist(A_1, A_2) > 0$, we have $$\diam(S_j) \geq \diam(S_j \cap A_1) + \diam(S_j \cap A_2).$$ The collections $\{S_j \cap A_1\}_{j \in \mathbb{N}}$ and $\{S_j \cap A_2\}_{j \in \mathbb{N}}$ form disjoint covers of $A_1$, $A_2$ respectively. So, when considering $\mathcal{H}_\delta^m(A_1 \cup A_2)$, it suffices to take the infimum over disjoint covers of $A_1$ and $A_2$... in fact, these infima are equal since we have reduced the size of our cover, and we use the fact that $ A \subseteq B \implies \inf(A) \geq \inf B$, which forces the following equality: 
			
			$$\inf_{\substack{A_1 \subseteq \bigcup S_j^1, A_2 \subseteq \bigcup S_k^2 \\ \boxed{\diam(S_j^1 \cup S_k^2) \leq \delta }  \\ S_j^1 \cap S_k^2 = \emptyset}} \sum_{j, k} \alpha_m  \left(\frac{\diam(S_j^1 \cup S_k^2)}{2}\right)^m = \inf_{\substack{(A_1 \cup A_2) \subseteq \bigcup S_j \\ \diam(S_j) \leq \delta}} \sum \alpha_m \left(\frac{\diam(S_j)}{2}\right)^m.$$ The result follows by passing to the limit as $\delta \to 0$. 
			
			\item[(3)]  Let $A \subseteq \mathbb{R}^n$ be arbitrary, and note that for any $S_j$ in a covering of $A$, we have $$\diam(S_j) = \diam(\overline{S_j}).$$ Since $\overline{S_j}$ is closed, it is Borel. We can write $$\mathcal{H}^m(A) = \lim_{\delta \to 0} \inf_{\substack{A \subseteq \bigcup S_j \\ \diam(S_j) < \delta}} \sum \alpha_m \left(\frac{\diam(S_j)}{2}\right)^m = \lim_{\delta \to 0} \inf_{\substack{A \subseteq \bigcup \overline{S_j} \\ \diam(\overline{S_j}) < \delta}} \sum \alpha_m \left(\frac{\diam(\overline{S_j})}{2}\right)^m.$$ 
			By a property of infimum, there exists a countable sequence of such Borel coverings $\{S_j^{(k)}\}_{k \in \mathbb{N}}$ defining $\mathcal{H}^m(A)$ (in the sense that their values when plugged into the summation above forms a non-increasing sequence of positive numbers tending to the value of $\mathcal{H}^m(A)$). Hence, let $$B = \bigcap_k \bigcup_j S_j^{(k)}$$ which has the same Hausdorff measure of $A$. Cool beans dude. 
		\end{itemize}
	\end{proof}

\subsection*{Hausdorff Measure Exercises}
	\noindent \textbf{Exercise 1.} Let $I$ be the unit interval $[0, 1]$ in $\mathbb{R}$. Prove that $\mathcal{H}^1(I) = 1$. 
	
	\noindent \textbf{Exercise 2.} Prove that $\mathcal{H}^n(\mathbb{B}^n(\mathbf{0}, 1)) < \infty$ just using the definition of Hausdorff measure.
	
	\noindent \textbf{Exercise 3.} Let $A$ be a nonempty subset of $\mathbb{R}^n$. First, prove that if $0 \leq m < k$ and $\mathcal{H}^m(A) < \infty$, then $\mathcal{H}^k(A) = 0$. 
	
	\noindent \textbf{Exercise 3.} Define a set $A \subset \mathbb{R}^2$ as follows: Let $A_0$ be a closed equilateral triangle of side 1. Let $A_1$ be the three equilateral triangular regions of side 1/3 in the corners of $A_0$. In general, let $A_{j + 1}$ be the triangular regions, a third of the size, in the corners of the triangles of $A_j$. Let $A = \bigcap A_j$. Prove that $\mathcal{H}^1(A) = 1$. 

\newpage 
\subsection*{The Hausdorff Dimension}
\newpage 


	\begin{definition}
		Let $A \subseteq \mathbb{R}^n$ be nonempty. The \textbf{\textit{Hausdorff dimension}} of $A$ is defined as 
			\begin{align*}
				\inf\{m \geq 0 : \mathcal{H}^m(A) < \infty\} &= \inf \{m : \mathcal{H}^m(A) = 0\} \\ 
				&= \sup \{m : \mathcal{H}^m(A) > 0\} \\
				&= \sup \{m : \mathcal{H}^m(A) = \infty\}.
			\end{align*}
	\end{definition}


	\begin{observation}
		All four definitions of the Hausdorff dimension above are equivalent.
	\end{observation}

\subsection*{Numerical Implementations [Big WIP]}
	\begin{itemize}
		\item For diameter...
			\begin{enumerate}
				\item Let $S = f(D)$ for some $D \subseteq \mathbb{R}^n$ and some function $f$. 
				
				\item Sample random points (the more points sampled, the more accurate the method) and take their distances. A stored array would probably look like (for defining a surface)
					
					\begin{table}[H]
						\centering 
						\begin{tabular}{c|c|c}
							$\mathbf{x}$ & $\mathbf{y}$ & $|(\mathbf{x}, f(\mathbf{x}) - (\mathbf{y}, f(\mathbf{y}))|$ \\
							\hline  
							(1, 2) & (3, 4) & 0.1234 \\
							(4, 3) & (1, 8) & 4.6512 \\
							$\vdots$ & $\vdots$ & $\vdots$ 
						\end{tabular}						
					\end{table}
				
				\item Then, $\diam(S)$ is reported as the max in the third column. Let's do it.
			\end{enumerate}
	\end{itemize}

	\begin{matlab}
		$\diam(S)$ numerical implementation for surfaces. 
	\end{matlab}
		\begin{proof}[Implementation]
			See the folder, will improve it later. 
		\end{proof}
\end{document}