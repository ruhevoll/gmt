\documentclass[10pt]{article}

\usepackage{amsmath, amssymb, amsthm, newtxtext, physics}
\usepackage{float}

\title{\textbf{The Hausdorff Measure}}
\date{}
\usepackage[margins = 0.25in]{geometry}
\theoremstyle{plain} 
\newtheorem{definition}{Definition}
\newtheorem{theorem}{Theorem}
\newtheorem{example}{Example}
\newtheorem{observation}{Observation}
\newtheorem{matlab}{MATLAB}
\DeclareMathOperator{\diam}{diam}
\begin{document}
	\maketitle 
	
In geometric measure theory, we like to work with outer measures so much that we just call them measures. 
	\begin{definition}
		A(n) (outer) \textbf{\textit{measure}} $\mu$ on $\mathbb{R}^n$ is a function $\mu: \mathcal{P}(\mathbb{R}^n) \to [0, + \infty]$ such that, for $\{A_i\}_{i \in \mathbb{N}}$ a countable collection of subsets of $\mathbb{R}^n$ and any $A \subseteq \bigcup_{i \in \mathbb{N}} A_i$, we have $$\mu(A) \leq \sum_{i \in \mathbb{N}} \mu(A_i).$$ A set $A \subseteq \mathbb{R}^n$ is \textbf{\textit{measurable}} if, for all $E \subset \mathbb{R}^n$, $$\mu(E \cap A) + \mu(E \cap A^C) = \mu(E).$$
	\end{definition}

	
\subsection*{The Hausdorff Measure}
	Some notation:	
		\begin{itemize}
			\item The \textbf{\textit{diameter}} of a set $S$ is denoted by $\diam(S)$ and is given by the following formula: $$\diam(S) = \sup \{|x - y| : x, y \in S\}.$$ 
			
			\item The Lebesgue measure of the closed unit ball $\mathbb{B}^m(0, 1) \subseteq \mathbb{R}^m$ is denoted by $\alpha_m$ and is given by the following formula $$\alpha_m = \frac{\pi^{m/2}}{\Gamma\left(\frac{m}{2} + 1\right)}.$$ 
		\end{itemize}
	\begin{definition}
		For any $A \subseteq \mathbb{R}^n$ define the $(\delta, m)$-\textbf{\textit{Hausdorff measure}} $\mathcal{H}_\delta^m(A)$ as $$\mathcal{H}_\delta^m(A) = \inf_{\substack{A \subset \bigcup S_j \\ \diam(S_j) \leq \delta}} \sum \alpha_m \left(\frac{\diam(S_j)}{2}\right)^m.$$ The \textbf{\textit{$m$-dimensional Hausdorff measure}} is defined as $\mathcal{H}^m(A) = \lim_{\delta \to 0} \mathcal{H}_{\delta}^m(A)$. 
	\end{definition}

	\begin{observation} ~
		Let $\mathcal{H}^m$ be the $m$-dimensional Hausdorff measure over $\mathbb{R}^n$. 
			\begin{itemize}
				\item[(1)] $\mathcal{H}^m$ is countably subadditive. 
				
				\item[(2)] All Borel sets of $\mathbb{R}^n$ are $\mathcal{H}^m$-measurable. 
				
				\item[(3)] For all $A \subset \mathbb{R}^n$ there exists a Borel subset $B \subset \mathbb{R}^n$ such that $$\mathcal{H}^m(A) = \mathcal{H}^m(B).$$
			\end{itemize}
	\end{observation}



\newpage 
\subsection*{The Hausdorff Dimension}
\newpage 


	\begin{definition}
		Let $A \subseteq \mathbb{R}^n$ be nonempty. The \textbf{\textit{Hausdorff dimension}} of $A$ is defined as 
			\begin{align*}
				\inf\{m \geq 0 : \mathcal{H}^m(A) < \infty\} &= \inf \{m : \mathcal{H}^m(A) = 0\} \\ 
				&= \sup \{m : \mathcal{H}^m(A) > 0\} \\
				&= \sup \{m : \mathcal{H}^m(A) = \infty\}.
			\end{align*}
	\end{definition}


	\begin{observation}
		All four definitions of the Hausdorff dimension above are equivalent.
	\end{observation}

\subsection*{Numerical Implementations [Big WIP]}
	\begin{itemize}
		\item For diameter...
			\begin{enumerate}
				\item Let $S = f(D)$ for some $D \subseteq \mathbb{R}^n$ and some function $f$. 
				
				\item Sample random points (the more points sampled, the more accurate the method) and take their distances. A stored array would probably look like (for defining a surface)
					
					\begin{table}[H]
						\centering 
						\begin{tabular}{c|c|c}
							$\mathbf{x}$ & $\mathbf{y}$ & $|(\mathbf{x}, f(\mathbf{x}) - (\mathbf{y}, f(\mathbf{y}))|$ \\
							\hline  
							(1, 2) & (3, 4) & 0.1234 \\
							(4, 3) & (1, 8) & 4.6512 \\
							$\vdots$ & $\vdots$ & $\vdots$ 
						\end{tabular}						
					\end{table}
				
				\item Then, $\diam(S)$ is reported as the max in the third column. Let's do it.
			\end{enumerate}
	\end{itemize}

	\begin{matlab}
		$\diam(S)$ numerical implementation for surfaces. 
	\end{matlab}
		\begin{proof}[Implementation]
			See the folder, will improve it later. 
		\end{proof}
\end{document}